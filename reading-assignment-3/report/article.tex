%%%%%%%%%%%%%%%%%%%%%%%%%%%%%%%%%%%%%%%%%
% Arsclassica Article
% LaTeX Template
% Version 1.1 (1/8/17)
%
% This template has been downloaded from:
% http://www.LaTeXTemplates.com
%
% Original author:
% Lorenzo Pantieri (http://www.lorenzopantieri.net) with extensive modifications by:
% Vel (vel@latextemplates.com)
%
% License:
% CC BY-NC-SA 3.0 (http://creativecommons.org/licenses/by-nc-sa/3.0/)
%
%%%%%%%%%%%%%%%%%%%%%%%%%%%%%%%%%%%%%%%%%

%----------------------------------------------------------------------------------------
%	PACKAGES AND OTHER DOCUMENT CONFIGURATIONS
%----------------------------------------------------------------------------------------

\documentclass[
10pt, % Main document font size
a4paper, % Paper type, use 'letterpaper' for US Letter paper
oneside, % One page layout (no page indentation)
%twoside, % Two page layout (page indentation for binding and different headers)
headinclude,footinclude, % Extra spacing for the header and footer
BCOR5mm, % Binding correction
]{scrartcl}

%%%%%%%%%%%%%%%%%%%%%%%%%%%%%%%%%%%%%%%%%
% Arsclassica Article
% Structure Specification File
%
% This file has been downloaded from:
% http://www.LaTeXTemplates.com
%
% Original author:
% Lorenzo Pantieri (http://www.lorenzopantieri.net) with extensive modifications by:
% Vel (vel@latextemplates.com)
%
% License:
% CC BY-NC-SA 3.0 (http://creativecommons.org/licenses/by-nc-sa/3.0/)
%
%%%%%%%%%%%%%%%%%%%%%%%%%%%%%%%%%%%%%%%%%

%----------------------------------------------------------------------------------------
%	REQUIRED PACKAGES
%----------------------------------------------------------------------------------------

\usepackage[
nochapters, % Turn off chapters since this is an article        
beramono, % Use the Bera Mono font for monospaced text (\texttt)
eulermath,% Use the Euler font for mathematics
pdfspacing, % Makes use of pdftex’ letter spacing capabilities via the microtype package
dottedtoc % Dotted lines leading to the page numbers in the table of contents
]{classicthesis} % The layout is based on the Classic Thesis style

\usepackage{arsclassica} % Modifies the Classic Thesis package

\usepackage[T1]{fontenc} % Use 8-bit encoding that has 256 glyphs

\usepackage[utf8]{inputenc} % Required for including letters with accents

\usepackage{graphicx} % Required for including images
\graphicspath{{Figures/}} % Set the default folder for images

\usepackage{enumitem} % Required for manipulating the whitespace between and within lists

\usepackage{lipsum} % Used for inserting dummy 'Lorem ipsum' text into the template

\usepackage{subfig} % Required for creating figures with multiple parts (subfigures)

\usepackage{amsmath,amssymb,amsthm} % For including math equations, theorems, symbols, etc

\usepackage{varioref} % More descriptive referencing

\usepackage{listings}
\usepackage{caption}
\usepackage{booktabs}

\usepackage{natbib}

%----------------------------------------------------------------------------------------
%	THEOREM STYLES
%---------------------------------------------------------------------------------------

\theoremstyle{definition} % Define theorem styles here based on the definition style (used for definitions and examples)
\newtheorem{definition}{Definition}

\theoremstyle{plain} % Define theorem styles here based on the plain style (used for theorems, lemmas, propositions)
\newtheorem{theorem}{Theorem}

\theoremstyle{remark} % Define theorem styles here based on the remark style (used for remarks and notes)

%----------------------------------------------------------------------------------------
%	HYPERLINKS
%---------------------------------------------------------------------------------------

\hypersetup{
%draft, % Uncomment to remove all links (useful for printing in black and white)
colorlinks=true, breaklinks=true, bookmarks=true,bookmarksnumbered,
urlcolor=webbrown, linkcolor=RoyalBlue, citecolor=webgreen, % Link colors
pdftitle={}, % PDF title
pdfauthor={\textcopyright}, % PDF Author
pdfsubject={}, % PDF Subject
pdfkeywords={}, % PDF Keywords
pdfcreator={pdfLaTeX}, % PDF Creator
pdfproducer={LaTeX with hyperref and ClassicThesis} % PDF producer
}

% Redefine caption for listings
\renewcommand\lstlistingname{Prompt}
\captionsetup[lstlisting]{labelsep=space, singlelinecheck=false, font={bf,footnotesize}, labelfont=bf}

\lstdefinestyle{myCustomStyle}{
  basicstyle=\ttfamily\small,
  frame=single,
  breaklines=true
  numbers=left,
  numberstyle=\tiny,
  keywordstyle=\color{blue},
  commentstyle=\color{gray},
}
 % Include the structure.tex file which specified the document structure and layout

\hyphenation{Fortran hy-phen-ation} % Specify custom hyphenation points in words with dashes where you would like hyphenation to occur, or alternatively, don't put any dashes in a word to stop hyphenation altogether

%----------------------------------------------------------------------------------------
%	TITLE AND AUTHOR(S)
%----------------------------------------------------------------------------------------

\title{\normalfont\spacedallcaps{Reading Assignment 3}} % The article title

% \subtitle{Critical Summary: The Winograd Schema Challenge} % Uncomment to display a subtitle

\author{\spacedlowsmallcaps{Caleb Moses*}} % The article author(s) - author affiliations need to be specified in the AUTHOR AFFILIATIONS block

\date{} % An optional date to appear under the author(s)

\renewcommand{\thesubsection}{\arabic{subsection}}
\makeatletter
\@addtoreset{subsection}{section}
\makeatother

%----------------------------------------------------------------------------------------

\begin{document}

%----------------------------------------------------------------------------------------
%	HEADERS
%----------------------------------------------------------------------------------------

\renewcommand{\sectionmark}[1]{\markright{\spacedlowsmallcaps{#1}}} % The header for all pages (oneside) or for even pages (twoside)
%\renewcommand{\subsectionmark}[1]{\markright{\thesubsection~#1}} % Uncomment when using the twoside option - this modifies the header on odd pages
\lehead{\mbox{\llap{\small\thepage\kern1em\color{halfgray} \vline}\color{halfgray}\hspace{0.5em}\rightmark\hfil}} % The header style

\pagestyle{scrheadings} % Enable the headers specified in this block

%----------------------------------------------------------------------------------------
%	TABLE OF CONTENTS & LISTS OF FIGURES AND TABLES
%----------------------------------------------------------------------------------------

\maketitle % Print the title/author/date block
\vspace{-20mm} % Adjust the value as needed

\setcounter{tocdepth}{2} % Set the depth of the table of contents to show sections and subsections only

%----------------------------------------------------------------------------------------
%	AUTHOR AFFILIATIONS
%----------------------------------------------------------------------------------------

\let\thefootnote\relax\footnotetext{* \textit{PhD Student, School of Computer Science, McGill University, Montreal, Canada}}

%----------------------------------------------------------------------------------------

%----------------------------------------------------------------------------------------
%	INTRODUCTION
%----------------------------------------------------------------------------------------

% Reading Assignment 3 - Outline
% COMP 550, Fall 2023
% Due Date: Nov 24, 2023 9:00 pm


% Introduction
%% \section{Introduction}
%% % Brief introduction to the paper and its relevance.
%% In the paper "The Wingrad Schema Challenge" by \citet{levesque2012winograd}, the authors propose an alternative to the Turing Test which aims to provide a practical means of determining whether a machine-based system such as a natural language generation algorithm is capable of thinking in the ``full-bodied sense we usually reserve for people''.

% Main Content
\section{Paper Summary}
% Your summary should include a description of the paper’s contents, its strengths and limitations, and any points that you did not understand. Since this paper discusses techniques that are not covered in the lectures, I expect that there will be concepts that you do not understand! Do not worry, and document this in your write-up.
The paper begins with a discussion of the Turing Test and its limitations. It then considered the general problem of recognizing textual entailment (RTE), based on the challenge by \citet{dagan2006pascal}, \citet{bobrow2007precision} and \citet{rus2008study}. It then outlines the Winograd Schema (WS) challenge, which it presents as an example of an RTE problem.

The paper outlines the required properties that a well-designed WS should have and discusses challenges and pitfalls in constructing them. It also makes arguments to support the claim that the solving WS is a sufficient condition for verifying that a machine system is capable of some degree of ``thinking''.

\subsection{Task Comparison}
% Compare the tasks of the turing test, recognizing textual entailment and the Winograd Schema challenge along any salient dimensions you seem important.
%% To solve the problem of verifying machine intelligence, Turing proposed the ``Imitation Game'', where a human interrogator conducts a long, free-flowing and unrestricted conversation with a machine after which they decide if the machine is distinguishable from a human on the basis of that conversation.

%% RTE is an alternative test whereby a subject is presented with yes-no questions concerning whether an English sentence entails another. The sentences are carefully designed such that they cover a broad range, english-speaking humans can pass it easily, it can be administered without expert judges and lastly that the sentences are difficult enough that being able to pass the test is considered evidence of thinking.

%% WS is a variant of RTE where the questions are designed such that the machine has to determine the referent of a pronoun or possessive adjective between two possible alternatives. The sentences are designed to be minimal, such that the answer appears obvious to a human.

Between the three problems, the Turing Test is considered undesirable in that it requires the machine to be deceptive towards the interrogator, while RTE and WS can be solved in earnest. RTE and WS also have the advantage that they can be conducted automatically, given a well constructed test dataset while the Turing Test is too open-ended for automated evaluation to be possible.

However, since RTE and WS require a degree of external knowledge in order to solve, the authors note cases where the answers might not be obvious enough or there could be disagreement between humans as to the answer depending on their background or other context.

\subsection{Dataset and motivation}

% The proposed dataset and task here are substantially motivated by \textit{how} systems might solve it. For example, WSC tests should be Google-proof. Why is this important? What are the advantages and disadvantages of motivating a task this way as opposed to by a downstream use case (e.g. machine translation or summarization)?

Because the WS task is motivated by how the systems might solve them, this has the advantage that solving it would be considered technical progress since existing methods will struggle to solve WS by its construction so it is clear new knowledge is needed. However, it is not clear that a system that could solve the WS challenge would be able to do much else beyond perhaps RTE, word sense disambiguation, co-reference resolution or similar computational linguistics problems.

\subsection{Can machines that solve WS ``think''?}

% Since the publication of the paper, it turns out that large language models trained on the Internet can solve WSC with high accuracy. Does this mean that LLMs can think “in the full-bodied sense we usually reserve for people”? Why or why not?

The strong claim the paper makes is that with a very high probability, anything that can answer the WS Challenge correctly is thinking in the ``full-bodied sense we usually reserve for people''. I think this is a stretch, since human thought implies many capabilities not tested in the WS challenge. One example is planning and system-2 thinking required to solve complex tasks. In this regard, Large Language Models are more powerful than would have been imagined by \citet{levesque2012winograd} when the challenge was designed.

The weaker claim of the paper is that with a very high probability, anything that answers correctly is engaging in behaviour that we would say shows thinking in people. I think this is more reasonable, because it is like saying that ``for a human to do this, they would have to think'' which is true.

%----------------------------------------------------------------------------------------
%	BIBLIOGRAPHY
%----------------------------------------------------------------------------------------

\renewcommand{\refname}{\spacedlowsmallcaps{References}} % For modifying the bibliography heading

\bibliographystyle{plainnat}

\bibliography{sample.bib} % The file containing the bibliography

%----------------------------------------------------------------------------------------

\end{document}

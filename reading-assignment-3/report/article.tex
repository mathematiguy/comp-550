%%%%%%%%%%%%%%%%%%%%%%%%%%%%%%%%%%%%%%%%%
% Arsclassica Article
% LaTeX Template
% Version 1.1 (1/8/17)
%
% This template has been downloaded from:
% http://www.LaTeXTemplates.com
%
% Original author:
% Lorenzo Pantieri (http://www.lorenzopantieri.net) with extensive modifications by:
% Vel (vel@latextemplates.com)
%
% License:
% CC BY-NC-SA 3.0 (http://creativecommons.org/licenses/by-nc-sa/3.0/)
%
%%%%%%%%%%%%%%%%%%%%%%%%%%%%%%%%%%%%%%%%%

%----------------------------------------------------------------------------------------
%	PACKAGES AND OTHER DOCUMENT CONFIGURATIONS
%----------------------------------------------------------------------------------------

\documentclass[
10pt, % Main document font size
a4paper, % Paper type, use 'letterpaper' for US Letter paper
oneside, % One page layout (no page indentation)
%twoside, % Two page layout (page indentation for binding and different headers)
headinclude,footinclude, % Extra spacing for the header and footer
BCOR5mm, % Binding correction
]{scrartcl}

\input{structure.tex} % Include the structure.tex file which specified the document structure and layout

\hyphenation{Fortran hy-phen-ation} % Specify custom hyphenation points in words with dashes where you would like hyphenation to occur, or alternatively, don't put any dashes in a word to stop hyphenation altogether

%----------------------------------------------------------------------------------------
%	TITLE AND AUTHOR(S)
%----------------------------------------------------------------------------------------

\title{\normalfont\spacedallcaps{Comp 550: Fall 2023\\Reading Assignment 2}} % The article title

\subtitle{Critical Summary: The Winograd Schema Challenge} % Uncomment to display a subtitle

\author{\spacedlowsmallcaps{Caleb Moses*}} % The article author(s) - author affiliations need to be specified in the AUTHOR AFFILIATIONS block

\date{} % An optional date to appear under the author(s)

%----------------------------------------------------------------------------------------

\begin{document}

%----------------------------------------------------------------------------------------
%	HEADERS
%----------------------------------------------------------------------------------------

\renewcommand{\sectionmark}[1]{\markright{\spacedlowsmallcaps{#1}}} % The header for all pages (oneside) or for even pages (twoside)
%\renewcommand{\subsectionmark}[1]{\markright{\thesubsection~#1}} % Uncomment when using the twoside option - this modifies the header on odd pages
\lehead{\mbox{\llap{\small\thepage\kern1em\color{halfgray} \vline}\color{halfgray}\hspace{0.5em}\rightmark\hfil}} % The header style

\pagestyle{scrheadings} % Enable the headers specified in this block

%----------------------------------------------------------------------------------------
%	TABLE OF CONTENTS & LISTS OF FIGURES AND TABLES
%----------------------------------------------------------------------------------------

\maketitle % Print the title/author/date block

\setcounter{tocdepth}{2} % Set the depth of the table of contents to show sections and subsections only

%----------------------------------------------------------------------------------------
%	AUTHOR AFFILIATIONS
%----------------------------------------------------------------------------------------

\let\thefootnote\relax\footnotetext{* \textit{PhD Student, School of Computer Science, McGill University, Montreal, Canada}}

%----------------------------------------------------------------------------------------

%----------------------------------------------------------------------------------------
%	INTRODUCTION
%----------------------------------------------------------------------------------------

% Reading Assignment 3 - Outline
% COMP 550, Fall 2023
% Due Date: Nov 24, 2023 9:00 pm

% Introduction
\section{Introduction}
% Brief introduction to the paper and its relevance.

% Main Content
\section{Summary of the Paper}
% A concise summary of the paper's main points and findings.

\subsection{Key Contributions}
% Discuss the primary contributions of the paper.

\subsection{Methodology}
% Briefly outline the methodology used in the study.

% Critical Analysis
\section{Critical Analysis}
% Your analysis of the paper, including strengths, limitations, and points of confusion.

\subsection{Strengths}
% Discuss the strong points and successes of the paper.

\subsection{Limitations}
% Identify and explain the limitations or weaknesses found in the paper.

\subsection{Points of Confusion}
% Mention any aspects that were not clear or fully understood.

% Comparative and Reflective Analysis
\section{Comparative and Reflective Analysis}
% Reflection on specific questions and comparison with related tasks.

\subsection{Comparative Dimensions}
% Compare the Winograd Schema Challenge with the Turing Test and Recognizing Textual Entailment.

\subsection{Dataset and Task Motivation}
% Reflect on the motivation behind the dataset and task, and its implications.

\subsection{Impact of Large Language Models}
% Discuss the implications of LLMs' success on WSC and the notion of machine intelligence.

% Conclusion
\section{Conclusion}
% Wrap up the summary with final thoughts and overall impression of the paper.

%----------------------------------------------------------------------------------------
%	BIBLIOGRAPHY
%----------------------------------------------------------------------------------------

\renewcommand{\refname}{\spacedlowsmallcaps{References}} % For modifying the bibliography heading

\bibliographystyle{plainnat}

\bibliography{sample.bib} % The file containing the bibliography

%----------------------------------------------------------------------------------------

\end{document}
